\chapter{Inleiding}
Het creëren van toegevoegde waarde met data is voor de meeste mkb’ers een grote uitdaging. Ze vragen zich letterlijk af “Ik heb data, en wat nu?”. We willen u helpen van de grond te komen door in kaart te brengen waar u nu staat en een plan te maken dat past bij uw ambities en middelen. Waar mogelijk kunnen we samen projecten oppakken binnen het Lectoraat Data Intelligence samen met studenten van de ICT academie en andere disciplines. Daarnaast bieden we een gesprek met een expert om u te helpen een plan te maken om uw organisatie intelligenter te maken met behulp van data. 

Deze vragenlijst is de eerste stap in het bepalen van het Data Intelligence Readiness level van uw organisatie. Door middel van deze zelfevaluatie krijgen we al een beeld van waar uw organisatie staat. Vervolgens is het mogelijk om op verzoek een gesprek met een expert te krijgen. Tijdens dit 1 op 1 vervolggesprek gaan we met deze zelfevaluatie als basis kijken waar de mogelijkheden liggen voor uw organisatie. Op deze manier kunnen we samen vaststellen wat de juiste vervolgstappen zijn voor uw organisatie.  
\chapter{Vervolgstappen}
\subsubsection{Stap 1: De zelfevaluatie via de Data Intelligence Readiness scan}

Stap 1 is gezet. Als u dit rapport leest heeft u de zelfevaluatie afgerond. Deze vragenlijst was de eerste stap in het creëren van toegevoegde waarde voor uw organisatie met data. Dankzij deze scan krijgen u en wij een beeld van waar volgens uzelf uw organisatie staat. Bijkomend voordeel is dat door het invullen van deze scan mogelijk een groter bewustzijn binnen uw organisatie ontstaan is als het gaat om het creëren van toegevoegde waarde met data. 

\subsubsection{Stap 2: 1 op 1 gesprek met een expert}

Als vervolg is het mogelijk om op verzoek een gesprek met een expert te krijgen. Deze volgende stap in onze reis is een 1 op 1 vervolggesprek met de zelfevaluatie uit stap 1 als vertrekpunt. In deze volgende stap gaan wij toetsen of de expert en u zich herkennen in het beeld dat is ontstaan op basis van de zelfevaluatie en dit waar mogelijk in overleg met u bijstellen. Samen zullen we in kaart brengen waar u nu staat, wat de ambities zijn en welke middelen beschikbaar zijn. Zo kunnen we identificeren waar de mogelijkheden liggen voor uw organisatie. Op deze manier kunnen we samen vaststellen wat de juiste vervolgstappen zijn in uw reis om toegevoegde waarde met data te creëren. Het uitgangspunt is hierbij altijd dat dit past bij waar u nu staat en waar u naartoe wilt als organisatie. 

\subsubsection{Stap 3: Concreet project in samenwerking met Lectoraat Data Intelligence}

Mogelijk kunnen we in stap 2 naast een routekaart ook een concreet vraagstuk formuleren waarmee we aan de slag kunnen gaan. Voor sommige mkb’ers betekent dit een eerste stap om kennis te maken met het creëren van toegevoegde waarde met data, voor anderen een verkenning om op termijn daadwerkelijk een implementatie te realiseren met een zakelijke partij. In overleg met de expert wordt een projectvoorstel opgesteld. Vervolgens wordt gekeken naar hoe de samenwerking met het Lectoraat Data Intelligence het beste kan worden vormgegeven. Samen met studenten en docenten gaan we dan aan de slag om u te helpen met het creëren van toegevoegde waarde op basis van data. 

\subsubsection{Vragen? Neem contact op:}

Tel: 045 400 6400 \newline
E-Mail: lectoraat.dataintelligence@zuyd.nl 